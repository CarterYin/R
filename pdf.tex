% 若编译失败,且生成 .synctex(busy) 辅助文件,可能有两个原因:
% 1. 需要插入的图片不存在:Ctrl + F 搜索 'figure' 将这些代码注释/删除掉即可
% 2. 路径/文件名含中文或空格:更改路径/文件名即可

% ------------------------------------------------------------- %
% >> ------------------ 文章宏包及相关设置 ------------------ << %
% 设定文章类型与编码格式
\documentclass[UTF8]{report}		

% 本文特殊宏包
    \usepackage{siunitx} % 埃米单位

% 本文的特殊宏定义
\def\Im{\mathrm{\,Im\,}}
\def\Re{\mathrm{\,Re\,}}
\def\Ln{\mathrm{\,Ln\,}}
\def\Arg{\mathrm{\,Arg\,}}
\def\Arccos{\mathrm{\,Arccos\,}}
\def\Arcsin{\mathrm{\,Arcsin\,}}
\def\Arctan{\mathrm{\,Arctan\,}}

% 通用宏定义
\def\N{\mathbb{N}}
\def\F{\mathbb{F}}
\def\Z{\mathbb{Z}}
\def\Q{\mathbb{Q}}
\def\R{\mathbb{R}}
\def\C{\mathbb{C}}
\def\T{\mathbb{T}}
\def\S{\mathbb{S}}
\def\A{\mathbb{A}}
\def\I{\mathscr{I}}
\def\d{\mathrm{d}}
\def\p{\partial}


% 导入基本宏包
    \usepackage[UTF8]{ctex}     % 设置文档为中文语言
    \usepackage[colorlinks, linkcolor=blue, anchorcolor=blue, citecolor=blue, urlcolor=blue]{hyperref}  % 宏包:自动生成超链接 (此宏包与标题中的数学环境冲突)
    % \usepackage{docmute}    % 宏包:子文件导入时自动去除导言区,用于主/子文件的写作方式,\include{./51单片机笔记}即可。注:启用此宏包会导致.tex文件capacity受限。
    \usepackage{amsmath}    % 宏包:数学公式
    \usepackage{mathrsfs}   % 宏包:提供更多数学符号
    \usepackage{amssymb}    % 宏包:提供更多数学符号
    \usepackage{pifont}     % 宏包:提供了特殊符号和字体
    \usepackage{extarrows}  % 宏包:更多箭头符号
    \usepackage{multicol}   % 宏包:支持多栏 
    \usepackage{graphicx}   % 宏包:插入图片
    \usepackage{float}      % 宏包:设置图片浮动位置
    %\usepackage{article}    % 宏包:使文本排版更加优美
    \usepackage{tikz}       % 宏包:绘图工具
    %\usepackage{pgfplots}   % 宏包:绘图工具
    \usepackage{enumerate}  % 宏包:列表环境设置
    \usepackage{enumitem}   % 宏包:列表环境设置
    \usepackage{listings}
    \usepackage{xcolor}

% 文章页面margin设置
    \usepackage[a4paper]{geometry}
        \geometry{top=1in}  % 1 inch= 2.46 cm, 0.75 inch = 1.85 cm
        \geometry{bottom=1in}
        \geometry{left=0.75in}
        \geometry{right=0.75in}   % 设置上下左右页边距
        \geometry{marginparwidth=1.75cm}    % 设置边注距离(注释、标记等)

% 配置数学环境
    \usepackage{amsthm} % 宏包:数学环境配置
    % theorem-line 环境自定义
        \newtheoremstyle{MyLineTheoremStyle}% <name>
            {11pt}% <space above>
            {11pt}% <space below>
            {}% <body font> 使用默认正文字体
            {}% <indent amount>
            {\bfseries}% <theorem head font> 设置标题项为加粗
            {:}% <punctuation after theorem head>
            {.5em}% <space after theorem head>
            {\textbf{#1}\thmnumber{#2}\ \ (\,\textbf{#3}\,)}% 设置标题内容顺序
        \theoremstyle{MyLineTheoremStyle} % 应用自定义的定理样式
        \newtheorem{LineTheorem}{Theorem.\,}
    % theorem-block 环境自定义
        \newtheoremstyle{MyBlockTheoremStyle}% <name>
            {11pt}% <space above>
            {11pt}% <space below>
            {}% <body font> 使用默认正文字体
            {}% <indent amount>
            {\bfseries}% <theorem head font> 设置标题项为加粗
            {:\\ \indent}% <punctuation after theorem head>
            {.5em}% <space after theorem head>
            {\textbf{#1}\thmnumber{#2}\ \ (\,\textbf{#3}\,)}% 设置标题内容顺序
        \theoremstyle{MyBlockTheoremStyle} % 应用自定义的定理样式
        \newtheorem{BlockTheorem}[LineTheorem]{Theorem.\,} % 使用 LineTheorem 的计数器
    % definition 环境自定义
        \newtheoremstyle{MySubsubsectionStyle}% <name>
            {11pt}% <space above>
            {11pt}% <space below>
            {}% <body font> 使用默认正文字体
            {}% <indent amount>
            {\bfseries}% <theorem head font> 设置标题项为加粗
            { \indent}% <punctuation after theorem head>
            {0pt}% <space after theorem head>
            {\textbf{#3}}% 设置标题内容顺序
        \theoremstyle{MySubsubsectionStyle} % 应用自定义的定理样式
        \newtheorem{definition}{}

%宏包:有色文本框(proof环境)及其设置
    \usepackage[dvipsnames,svgnames]{xcolor}    %设置插入的文本框颜色
    \usepackage[strict]{changepage}     % 提供一个 adjustwidth 环境
    \usepackage{framed}     % 实现方框效果
        \definecolor{graybox_color}{rgb}{0.95,0.95,0.96} % 文本框颜色。修改此行中的 rgb 数值即可改变方框纹颜色,具体颜色的rgb数值可以在网站https://colordrop.io/ 中获得。(截止目前的尝试还没有成功过,感觉单位不一样)(找到喜欢的颜色,点击下方的小眼睛,找到rgb值,复制修改即可)
        \newenvironment{graybox}{%
        \def\FrameCommand{%
        \hspace{1pt}%
        {\color{gray}\small \vrule width 2pt}%
        {\color{graybox_color}\vrule width 4pt}%
        \colorbox{graybox_color}%
        }%
        \MakeFramed{\advance\hsize-\width\FrameRestore}%
        \noindent\hspace{-4.55pt}% disable indenting first paragraph
        \begin{adjustwidth}{}{7pt}%
        \vspace{2pt}\vspace{2pt}%
        }
        {%
        \vspace{2pt}\end{adjustwidth}\endMakeFramed%
        }

% 外源代码插入设置
    % matlab 代码插入设置
    %\usepackage{matlab-prettifier}
    %    \lstset{
    %        style=Matlab-editor,  % 继承matlab代码颜色等
    %    }
    %\usepackage[most]{tcolorbox} % 引入tcolorbox包 
    %\usepackage{listings} % 引入listings包
    %    \tcbuselibrary{listings, skins, breakable}
    %    \newfontfamily\codefont{Consolas} % 定义需要的 codefont 字体
    %    \lstdefinestyle{matlabstyle}{
    %        language=Matlab,
    %        basicstyle=\small\ttfamily\codefont,    % ttfamily 确保等宽 
    %        breakatwhitespace=false,
    %        breaklines=true,
    %        captionpos=b,
    %        keepspaces=true,
    %        numbers=left,
    %        numbersep=15pt,
    %        showspaces=false,
    %        showstringspaces=false,
    %        showtabs=false,
    %        tabsize=2
    %    }
    %    \newtcblisting{matlablisting}{
    %        arc=2pt,        % 圆角半径
    %        top=-5pt,
    %        bottom=-5pt,
    %        left=1mm,
    %        listing only,
    %        listing style=matlabstyle,
    %        breakable,
    %        colback=white   % 选一个合适的颜色
    %    }

    \definecolor{codegreen}{rgb}{0,0.6,0}
    \definecolor{codegray}{rgb}{0.5,0.5,0.5}
    \definecolor{codepurple}{rgb}{0.58,0,0.82}
    \definecolor{backcolour}{rgb}{0.95,0.95,0.92}

    \lstdefinestyle{mystyle}{
    backgroundcolor=\color{backcolour},   
    commentstyle=\color{codegreen},
    keywordstyle=\color{magenta},
    numberstyle=\tiny\color{codegray},
    stringstyle=\color{codepurple},
    basicstyle=\ttfamily\footnotesize,
    breakatwhitespace=false,         
    breaklines=true,                 
    captionpos=b,                    
    keepspaces=true,                 
    numbers=left,                    
    numbersep=5pt,                  
    showspaces=false,                
    showstringspaces=false,
    showtabs=false,                  
    tabsize=2
    }

    \lstset{style=mystyle}

% table 支持
    \usepackage{booktabs}   % 宏包:三线表
    \usepackage{tabularray} % 宏包:表格排版
    \usepackage{longtable}  % 宏包:长表格


%figure 设置
%    \usepackage{graphicx}  % 支持 jpg, png, eps, pdf 图片 
%    \usepackage{svg}       % 支持 svg 图片
%        \svgsetup{
%             指向 inkscape.exe 的路径
%            inkscapeexe = C:/aa_MySame/inkscape/bin/inkscape.exe, 
%            inkscapeexe = C:/aa_MySame/inkscape/bin/inkscape.exe, 
%             一定程度上修复导入后图片文字溢出几何图形的问题
%            inkscapelatex = false                 
%        }
%    \usepackage{subcaption} % subfigure 子图支持

%图表进阶设置
%    \usepackage{caption}    % 图注、表注
%        \captionsetup[figure]{name=图}  
%        \captionsetup[table]{name=表}
%        \captionsetup{labelfont=bf, font=small}
%    \usepackage{float}     % 图表位置浮动设置 

% 圆圈序号自定义
    \newcommand*\circled[1]{\tikz[baseline=(char.base)]{\node[shape=circle,draw,inner sep=0.8pt, line width = 0.03em] (char) {\small \bfseries #1};}}   % TikZ solution

% 列表设置
%    \usepackage{enumitem}   % 宏包:列表环境设置
%        \setlist[enumerate]{itemsep=0pt, parsep=0pt, topsep=0pt, partopsep=0pt, leftmargin=3.5em} 
%        \setlist[itemize]{itemsep=0pt, parsep=0pt, topsep=0pt, partopsep=0pt, leftmargin=3.5em}
%        \newlist{circledenum}{enumerate}{1} % 创建一个新的枚举环境  
%        \setlist[circledenum,1]{  
%            label=\protect\circled{\arabic*}, % 使用 \arabic* 来获取当前枚举计数器的值,并用 \circled 包装它  
%            ref=\arabic*, % 如果需要引用列表项,这将决定引用格式(这里仍然使用数字)
%            itemsep=0pt, parsep=0pt, topsep=0pt, partopsep=0pt, leftmargin=3.5em
%        }  

% 其它设置
    % 脚注设置
        \renewcommand\thefootnote{\ding{\numexpr171+\value{footnote}}}
    % 参考文献引用设置
        \bibliographystyle{unsrt}   % 设置参考文献引用格式为unsrt
        \newcommand{\upcite}[1]{\textsuperscript{\cite{#1}}}     % 自定义上角标式引用
    % 文章序言设置
        \newcommand{\cnabstractname}{序言}
        \newenvironment{cnabstract}{%
            \par\Large
            \noindent\mbox{}\hfill{\bfseries \cnabstractname}\hfill\mbox{}\par
            \vskip 2.5ex
            }{\par\vskip 2.5ex}

% 文章默认字体设置
    \usepackage{fontspec}   % 宏包:字体设置
        \setCJKmainfont{SimSun}    % 设置中文字体为宋体字体
        \setCJKmainfont[AutoFakeBold=3]{SimSun} % 设置加粗字体为 SimSun 族,AutoFakeBold 可以调整字体粗细
        \setmainfont{Times New Roman} % 设置英文字体为Times New Roman

% 各级标题自定义设置
    \usepackage{titlesec}   
        \titleformat{\chapter}[hang]{\normalfont\Large\bfseries\centering}{}{20pt}{}
        \titlespacing*{\chapter}{0pt}{-20pt}{20pt} % 控制上方空白的大小
        % section标题自定义设置 
        \titleformat{\section}[hang]{\normalfont\Large\bfseries}{§\,\thesection\,}{8pt}{}
        % subsubsection标题自定义设置
        %\titleformat{\subsubsection}[hang]{\normalfont\bfseries}{}{8pt}{}

% >> ------------------ 文章宏包及相关设置 ------------------ << %
% ------------------------------------------------------------- %

% ----------------------------------------------------------- %
% >> --------------------- 文章信息区 --------------------- << %
% 页眉页脚设置
    \usepackage{fancyhdr}   %宏包:页眉页脚设置
        \pagestyle{fancy}
        \fancyhf{}
        \cfoot{\thepage}
        \renewcommand\headrulewidth{1pt}
        \renewcommand\footrulewidth{0pt}
        \lhead{Carter Yin} 
        \chead{R Review}    
        \rhead{yinchao050313@gmail.com}
%文档信息设置
    \title{R Review}
    \author{尹超\\ \footnotesize 中国科学院大学,北京 100049\\ Carter Yin \\ \footnotesize University of Chinese Academy of Sciences, Beijing 100049, China}
    \date{\footnotesize 2024.8 -- 2025.1}
% >> --------------------- 文章信息区 --------------------- << %
% ----------------------------------------------------------- %

% 开始编辑文章

\begin{document} 
\zihao{5}             % 设置全文字号大小, -4 为小四, 5 为五号

% --------------------------------------------------------------- %
% >> --------------------- 封面序言与目录 --------------------- << %
% 封面
    \maketitle\newpage   % 生成封面并换页
    \pagenumbering{Roman} % 页码为大写罗马数字
    \thispagestyle{fancy}   % 显示页码、页眉等

% 序言
    \begin{cnabstract}\normalsize 
        R  Review
    \end{cnabstract}    
\addcontentsline{toc}{chapter}{序言} % 手动添加为目录

% 目录
    \setcounter{tocdepth}{4}                % 目录深度(为1时显示到section)
    \tableofcontents                        % 目录页
    \addcontentsline{toc}{chapter}{目录}    % 手动添加此页为目录
    \thispagestyle{fancy}                   % 显示页码、页眉等 

% 收尾工作
    \newpage    
    \pagenumbering{arabic} 


    
% >> --------------------- 封面序言与目录 --------------------- << %
% --------------------------------------------------------------- %

\chapter{问题 1:使用最大似然估计 (MLE) 进行自助法(Bootstrapping)}

\begin{lstlisting}[language=R]
    # 定义样本大小(观察值的数量)
    mysamplesize = 115
    
    # 构造置信区间时的显著性水平(alpha)
    alpha = 0.05
    
    # 定义真实参数值(指数分布的偏移量)
    mytheta = 1
    
    # 从指数分布生成一个大小为 `mysamplesize` 的随机样本,并加上偏移量 `mytheta`
    mysample = rexp(mysamplesize, rate = 1) + mytheta
    
    # 计算最大似然估计(MLE),这里采用排序后样本中的最小值(指数分布的MLE就是样本中的最小值)
    mymle = sort(mysample)[1]
    
    # 自助法样本的数量
    bootstrapsize = 1000
    
    # 初始化一个向量来存储每次自助法估计的结果
    bootstrapestimates = rep(0, bootstrapsize)
    
    # 执行自助法:生成自助样本并计算它们的估计值
    for (ii in 1:bootstrapsize) {
        # 使用MLE作为偏移量生成一个新的自助样本
        bootstrapsample = rexp(mysamplesize, rate = 1) + mymle
        
        # 每个自助样本的估计值是排序后的最小值(类似MLE)
        bootstrapestimates[ii] = sort(bootstrapsample)[1]
    }
    
    # 排序自助法的估计值,为计算分位数做准备
    bootstrapquantiles = sort(bootstrapestimates - mymle)
    
    # 计算置信区间的下分位数(alpha/2)
    lowerquantile = bootstrapquantiles[round(bootstrapsize*alpha*0.5)]
    
    # 计算置信区间的上分位数(1-alpha/2)
    upperquantile = bootstrapquantiles[round(bootstrapsize*(1-alpha*0.5))]
    
    # 计算置信区间的下界,通过从MLE中减去上分位数
    lowerbound = mymle - upperquantile
    
    # 计算置信区间的上界,通过从MLE中减去下分位数
    upperbound = mymle - lowerquantile
    
    # 输出置信区间的下界
    lowerbound
    
    # 输出置信区间的上界
    upperbound

> lowerbound
[1] 0.9730528
> upperbound
[1] 1.003367
    
\end{lstlisting}




\chapter{问题 2:使用矩估计进行自助法(Bootstrapping)}


\begin{lstlisting}[language=R]
    # 样本大小与问题1相同
mysamplesize = 115

# 构造置信区间时的显著性水平(alpha)
alpha = 0.05

# 定义真实参数值(指数分布的偏移量)
mytheta = 1

# 从指数分布生成一个样本,并加上偏移量 `mytheta`
mysample = rexp(mysamplesize, rate = 1) + mytheta

# 计算矩估计,这里是样本均值减去1(对于指数分布,矩估计是样本均值减去1)
mymomentestimation = mean(mysample) - 1

# 自助法样本的数量
bootstrapsize = 1000

# 初始化一个向量来存储每次自助法估计的结果
bootstrapestimates = rep(0, bootstrapsize)

# 执行自助法:生成自助样本并计算它们的估计值
for (ii in 1:bootstrapsize) {
    # 使用矩估计作为偏移量生成一个新的自助样本
    bootstrapsample = rexp(mysamplesize, rate = 1) + mymomentestimation
    
    # 每个自助样本的矩估计值是样本均值减去1
    bootstrapestimates[ii] = mean(bootstrapsample) - 1
}

# 排序自助法的估计值,为计算分位数做准备
bootstrapquantiles = sort(bootstrapestimates - mymomentestimation)

# 计算置信区间的下分位数(alpha/2)
lowerquantile = bootstrapquantiles[round(bootstrapsize*alpha*0.5)]

# 计算置信区间的上分位数(1-alpha/2)
upperquantile = bootstrapquantiles[round(bootstrapsize*(1-alpha*0.5))]

# 计算置信区间的下界,通过从矩估计中减去上分位数
lowerbound = mymomentestimation - upperquantile

# 计算置信区间的上界,通过从矩估计中减去下分位数
upperbound = mymomentestimation - lowerquantile

# 输出置信区间的下界
lowerbound

# 输出置信区间的上界
upperbound


> lowerbound
[1] 0.8166889
> upperbound
[1] 1.165083


\end{lstlisting}


\section*{解释}

\subsection*{随机样本生成}

\texttt{rexp(mysamplesize, rate = 1)} 用于从指数分布中生成随机数,\texttt{rate = 1} 指的是指数分布的速率参数为 1。在两种情况中,都将 \texttt{mytheta}(MLE 中的真实值)或 \texttt{mymomentestimation}(矩估计中的真实值)加到样本中。

\subsection*{最大似然估计 (MLE)}

对于指数分布,MLE 是排序后样本中的最小值(\texttt{sort(mysample)[1]}),因为指数分布的 MLE 就是样本中的最小值。

\subsection*{矩估计}

矩估计是计算样本均值减去 1(\texttt{mean(mysample) - 1}),这是根据矩估计法得到的结果,用于指数分布。

\subsection*{自助法(Bootstrapping)}

自助法通过对样本进行有放回抽样,再次计算估计值(MLE 或矩估计),并计算这些估计值的分布。置信区间是基于自助法估计值的分位数计算的(\texttt{lowerquantile} 和 \texttt{upperquantile})。

\subsection*{置信区间计算}

最终步骤是通过自助法估计值的分位数计算置信区间的上下界。计算方式是将分位数从原始估计值(MLE 或矩估计)中减去。

这个代码实现了通过自助法估计最大似然估计(MLE)和矩估计的置信区间。两者的逻辑类似,但初始估计的计算方法不同。







\chapter{问题3:单样本Z检验(已知方差)}


\begin{lstlisting}[language=R]
 # one-sample z-test (variance known) ---------------------------------------------

    nail = c(2.942371, 2.988662, 3.106234, 3.109316, 3.118427, 3.132254, 
         3.140042, 3.170188, 2.902562, 3.128003, 3.146441, 2.978240, 
         3.103600, 3.003394, 3.044384, 2.849916)  # 样本数据
sigma = 0.1  # 方差已知,sigma为已知的标准差
mu = 3  # 零假设下的均值
n = length(nail)  # 样本数量
# 检验统计量,零假设下服从标准正态分布
Z_statistic = sqrt(n) * (mean(nail) - mu) / sigma  # 计算Z统计量
# help(pnorm)  # 查询pnorm函数的帮助文档
p_value = 2 * pnorm(Z_statistic, lower.tail = F)  # 计算双尾p值
p_value  # 输出p值
# 0.03076609  # 计算得到的p值

# 方法2:使用BSDA包进行Z检验
# install.packages("BSDA")  # 安装BSDA包
library(BSDA)  # 加载BSDA包
# help(z.test)  # 查询z.test函数的帮助文档
result = z.test(nail, mu = mu, alternative = c("two.sided"), sigma.x = sigma)  # 使用z.test进行Z检验
p_value = result$p.value  # 提取p值
p_value  # 输出p值
# 0.03076609  # 计算得到的p值

### 单样本t检验(未知方差)

# help(t.test)  # 查询t.test函数的帮助文档
result = t.test(nail, mu = mu, alternative = c("two.sided"))  # 进行单样本t检验
p_value = result$p.value  # 提取p值
p_value  # 输出p值
# 0.04297243  # 计算得到的p值

### 双样本t检验

# H0:磷肥没有效果 mu1=mu2
# H1:磷肥有正面效果(提高产量) mu1-mu2>0
# 设置两组数据
x = c(62, 57, 65, 60, 63, 58, 57, 60, 60, 58)  # 第一组数据
y = c(50, 59, 56, 57, 58, 57, 56, 55, 57)  # 第二组数据
# 假设两个总体方差相等,进行双样本t检验
result = t.test(x, y, alternative = c("greater"), var.equal = T)  # 方差相等假设
p_value = result$p.value  # 提取p值
p_value  # 输出p值
# 0.002471312  # 计算得到的p值

# 假设两个总体方差不相等,进行双样本t检验
result = t.test(x, y, alternative = c("greater"), var.equal = F)  # 方差不等假设
p_value = result$p.value  # 提取p值
p_value  # 输出p值
# 0.002450701  # 计算得到的p值

### 检验

# 不使用函数计算p值

# help(pt)  # 查询pt函数的帮助文档

# 单样本Z检验(已知方差)
Z_statistic <- 4 * (mean(nail) - 3) / 0.1  # 计算Z统计量
p_value <- 2 * pnorm(Z_statistic, lower.tail = F)  # 计算双尾p值
p_value  # 输出p值

# 单样本t检验(未知方差)
S <- sqrt(sum((nail - mean(nail))^2) / 15)  # 计算样本的标准差
T_statistic <- 4 * (mean(nail) - 3) / S  # 计算t统计量
p_value <- 2 * pt(T_statistic, df = 15, lower.tail = F)  # 计算双尾p值
p_value  # 输出p值

# 双样本t检验(方差相等)
S1 <- sqrt(sum((x - mean(x))^2) / 9)  # 计算第一组样本的标准差
S2 <- sqrt(sum((y - mean(y))^2) / 8)  # 计算第二组样本的标准差
T_statistic <- (mean(x) - mean(y)) / sqrt(1/9 + 1/10) / sqrt((9 * S1^2 + 8 * S2^2) / 17)  # 计算t统计量
p_value <- pt(T_statistic, df = 17, lower.tail = F)  # 计算p值
p_value  # 输出p值

# 双样本t检验(方差不等)
SEw <- sqrt(1/10 * S1^2 + 1/9 * S2^2)  # 计算标准误差
df <- SEw^4 / (S1^4/10/10/9 + S2^4/9/9/8)  # 计算自由度
T_statistic <- (mean(x) - mean(y)) / SEw  # 计算t统计量
p_value <- pt(T_statistic, df = df, lower.tail = F)  # 计算p值
p_value  # 输出p值


\end{lstlisting}



\section*{解释}

\subsection*{单样本 Z 检验(已知方差)}

使用已知的总体标准差进行 Z 检验,通过计算 Z 统计量(\texttt{Z\_statistic})并从标准正态分布中查找 p 值。

\subsection*{方法 2:BSDA 包的 Z 检验}

使用 \texttt{BSDA} 包中的 \texttt{z.test} 函数来进行 Z 检验,并提取 p 值。

\subsection*{单样本 t 检验(未知方差)}

在样本方差未知的情况下,使用 t 检验并从结果中提取 p 值。

\subsection*{双样本 t 检验}

通过假设方差相等或不等,进行双样本 t 检验,检验两个组的均值差异。

\subsection*{检验部分}

不使用内建函数手动计算 Z 统计量、t 统计量和 p 值,分别进行了单样本和双样本检验的计算过程。














\chapter{问题4:Wilcoxon秩和检验、t检验和Hardy-Weinberg平衡检验}

\begin{lstlisting}[language=R]
    Wilcoxon 检验
    # help(wilcox.test)  # 查询 wilcox.test 函数的帮助文档
    Aarea = c(62, 57.04, 65, 60, 63, 58.1, 57.001, 60.1, 60.2, 58)  # 样本A数据
    Barea = c(50, 59, 56.01, 57.01, 58.05, 57.02, 56, 55, 57.03)  # 样本B数据
    result = wilcox.test(Aarea, Barea, exact = TRUE, alternative = c("two.sided"))  # 进行Wilcoxon秩和检验(双尾检验)
    p_value = result$p.value  # 提取p值
    p_value  # 输出p值
    # p_value = 0.002987724  # 计算得到的p值

    t 检验
    result = t.test(Aarea, Barea, alternative = c("greater"))  # 进行t检验(单尾检验,假设样本A的均值大于样本B的均值)
    p_value = result$p.value  # 提取p值
    p_value  # 输出p值
    # p_value = 0.002307728  # 计算得到的p值

    Hardy-Weinberg 平衡检验
    # MLE of theta
    theta <- 874/(1184+874)  # 计算最大似然估计(MLE)的theta值,这里假设是基因型频率
    # MLE of p_M
    p1 <- (1-theta)^2  # MLE of p_M,计算纯合隐性基因型(MM)的频率
    # MLE of p_MN
    p2 <- 2*theta*(1-theta)  # MLE of p_MN,计算杂合基因型(MN)的频率
    # MLE of p_N
    p3 <- theta^2  # MLE of p_N,计算纯合显性基因型(NN)的频率
    # expression of test statistic
    Z <- (342-1029*p1)^2/(1029*p1) + (500-1029*p2)^2/(1029*p2) + (187-1029*p3)^2/(1029*p3)  # 计算卡方检验统计量
    p_value <- pchisq(Z, df = 1, lower.tail = FALSE)  # 计算卡方分布的p值,df=1是自由度
    p_value  # 输出p值
    # p_value = 0.8569258  # 计算得到的p值


  
\end{lstlisting}





\section*{解释}

\subsection*{Wilcoxon 检验}

\texttt{wilcox.test()} 用于执行非参数检验,比较两组数据的分布是否相同。此处为双尾检验。通过 \texttt{result\$p.value} 提取 p 值,结果表明 p 值为 0.002987724,表明两组数据差异显著。

\subsection*{t 检验}

\texttt{t.test()} 用于执行传统的 t 检验,这里使用单尾检验,假设组 A 的均值大于组 B。通过 \texttt{result\$p.value} 提取并输出 t 检验的 p 值,结果为 0.002307728,表明组 A 的均值显著大于组 B。

\subsection*{Hardy-Weinberg 平衡检验}

使用最大似然估计(MLE)方法计算等位基因频率,并基于这些频率计算卡方检验的统计量。通过 \texttt{pchisq()} 计算卡方分布的 p 值,结果为 0.8569258,表示未拒绝零假设,基因频率符合 Hardy-Weinberg 平衡。










\chapter{问题5}

\begin{lstlisting}[language=R]
    Jane Austen 数据分析
    table1 <- matrix(c(14, 133, 12, 241, 11, 259, 16, 180, 14, 285, 6, 265, 8, 93, 
                   12, 139, 8, 221, 2, 81, 1, 153, 17, 204), 
                 nrow = 6)  # 创建一个6行4列的矩阵table1,包含Jane Austen数据
   result <- chisq.test(table1[, 1:3])  # 对table1的前三列进行卡方检验
   p_value <- result$p.value  # 提取p值
   p_value  # 输出p值
   # p_value = 0.009734904  # 计算得到的p值

   # 前三列求行和作为一列
   table2 <- matrix(c(38, 406, 38, 665, 25, 745, 2, 81, 1, 153, 17, 204), nrow = 6)  # 创建一个新的6行4列矩阵table2
   result <- chisq.test(table2)  # 对table2进行卡方检验
   p_value <- result$p.value  # 提取p值
   p_value  # 输出p值
   # p_value = 1.669517e-05  # 计算得到的p值

   探索性数据分析(EDA)
   library(MASS)  # 加载MASS包,提供统计工具和数据集
   library(lattice)  # 加载lattice包,提供绘图功能

   # base:plot, ggplot2, lattice  # 介绍常用的绘图工具包

   # R内置数据集hills,包含1984年35场苏格兰山地赛跑的记录时间数据
   hills  # 查看hills数据
   ?hills  # 查看hills数据集的帮助文档,了解数据集的背景和描述

   splom(~ hills)  # 散点图矩阵,展示数据中各个变量的关系

   attach(hills)  # 使用attach函数,使数据集hills中的变量可以直接使用

   plot(hills$dist, hills$time)  # 绘制散点图,展示distance(距离)与time(时间)的关系

   # interactive()测试当前是否是R的交互模式
   # 如果是,则运行identify函数打开一个交互窗口
   # 该交互窗口可以在图上的点上直接点击,显示标签
   # 在原版R中可以正常操作,但是Rstudio里出现了偏差,原因不明
   # if(interactive()) identify(dist, time, row.names(hills))  # 在交互模式下显示数据点的标签

   # detach(hills)  # 取消attach,避免之后引用变量时出现混淆

   4.5 Trellis 图形
   xyplot(time ~ dist, data = hills,  # 创建一个Trellis散点图,展示时间与距离的关系
       panel = function(x, y, ...) {  # 定义绘图面板函数
           panel.xyplot(x, y, ...)  # 绘制散点图
           panel.lmline(x, y, type = "l")  # 添加最小二乘回归线
           panel.abline(lqs(y ~ x), lty = 3)  # 添加resistant regression线(稳健回归线),lty控制线条类型
       }
)

median(hills$time)  # 计算时间数据的中位数
mad(hills$time)  # 计算时间数据的绝对中位差
histogram(~time, data = hills)  # 绘制时间数据的直方图
densityplot(~time, data = hills)  # 绘制时间数据的密度图
stem(hills$time)  # 绘制时间数据的茎叶图
qqnorm(hills$time)  # 绘制时间数据的QQ图,用于检验数据是否符合正态分布


Michelson 实验数据
michelson  # 查看michelson数据集,包含光速实验数据
?michelson  # 查看michelson数据集的帮助文档

bwplot(Expt ~ Speed, data = michelson, main = "Speed of Light Data", ylab = "Experiment No.")  # 绘制箱线图,展示不同实验光速的分布
# title("Speed of Light Data")  # 添加标题,但与lattice包的绘图函数一起使用时可能会出现问题



\end{lstlisting}

    
\section*{解释}

\subsection*{Jane Austen 数据分析}

使用 \texttt{chisq.test()} 进行卡方检验,比较不同类别的数据频率。创建了两个数据表 \texttt{table1} 和 \texttt{table2},并对其进行卡方检验,提取并输出 p 值,判断数据是否存在显著差异。

\subsection*{探索性数据分析(EDA)}

使用 \texttt{MASS} 和 \texttt{lattice} 包进行数据分析和可视化。展示了如何使用散点图、散点图矩阵、Trellis 图等可视化方法探索数据,帮助理解数据分布和关系。计算了数据的中位数、绝对中位差,并绘制了直方图、密度图、茎叶图和 QQ 图。

\subsection*{Michelson 实验数据}

使用 \texttt{michelson} 数据集,绘制光速实验的箱线图,展示不同实验中光速测量的分布。



\chapter{12.02 数据处理与可视化}

\begin{lstlisting}[language=R]
    # 请注意R的工作目录和Pearson.csv文件所在的目录

# 获取当前工作目录
getwd()  
# 设置工作目录,确保R能找到文件
setwd("C:/Users/WangKe/Desktop/TA/2024TA/exercise class")  
# 读取CSV文件“Pearson.csv”,并将数据存储在height中
height = read.csv("Pearson.csv")  
# 如果文件是txt格式的,可以用read.table方法读取
# height = read.table("Pearson.txt", header = TRUE)

# 另一种方法:使用绝对路径直接读取CSV文件
height = read.csv("C:/Users/WangKe/Desktop/TA/2024TA/exercise class/Pearson.csv")  
# 另一种方法,读取txt文件
# height = read.table("C:/Users/WangKe/Desktop/TA/2024TA/exercise class/Pearson.txt", header = TRUE)

# 绘制散点图,bty控制图的边框类型,pch控制点的形状
plot(Son ~ Father, data = height, bty = "l", pch = 20)  
# 添加直线,a为截距,b为斜率,lty控制线条类型,lwd控制线条宽度
abline(a = 0, b = 1, lty = 1, lwd = 2)  
# 画出线性回归模型的回归线,Son为y,Father为x
abline(lm(Son ~ Father, data = height), lty = 1, lwd = 2)  

# 计算Father列的标准差
Father.sd = sd(height$Father)  
# 计算Son列的标准差
Son.sd = sd(height$Son)  
# 计算Father列的均值
Father.ave = mean(height$Father)  
# 计算Son列的均值
Son.ave = mean(height$Son)  

# 计算Pearson相关系数,衡量Father和Son之间的线性关系
Pearson.cor = cor(height$Father, height$Son)  

# 绘制Father数据的直方图,检查其分布情况
hist(height$Father)  
# 绘制Son数据的直方图,检查其分布情况
hist(height$Son)  

# 绘制Father数据的核密度图,进一步查看分布
plot(density(height$Father))  
# 绘制Son数据的核密度图,进一步查看分布
plot(density(height$Son))  

# 绘制QQ图,检查Father数据是否服从正态分布
qqnorm(height$Father)  
qqline(height$Father)  # 绘制QQ图的参考线
# 绘制QQ图,检查Son数据是否服从正态分布
qqnorm(height$Son)  
qqline(height$Son)  # 绘制QQ图的参考线

\end{lstlisting}

\section*{12.02}

\subsection*{读取数据}

通过 \texttt{read.csv()} 或 \texttt{read.table()} 函数读取 CSV 或 TXT 文件,并设置工作目录。

\subsection*{可视化}

使用 \texttt{plot()} 绘制散点图,\texttt{abline()} 添加回归线,\texttt{lm()} 进行线性回归。计算数据的标准差、均值和 Pearson 相关系数。

\subsection*{分布检验}

通过直方图、核密度图和 QQ 图检查数据的分布情况,检查是否符合正态分布。




\chapter{12.04 数据处理与回归分析}

\begin{lstlisting}[language=R]
    # 设置工作目录,确保R能找到文件
    setwd("C:/Users/WangKe/Desktop/TA/2024TA/exercise class")  
    # 读取CSV文件“Pearson.csv”,并将数据存储在height中
    height = read.csv("Pearson.csv")  
    # 绘制Son和Father的散点图
    plot(Son ~ Father, data = height, bty = "l", pch = 20)  
    
    # 计算Son和Father的均值
    Son.ave = mean(height$Son)  
    Father.ave = mean(height$Father)  
    # 绘制Son和Father的回归线,截距为Son的均值减去Father的均值,斜率为1
    abline(a = Son.ave - Father.ave, b = 1, lty = 1, lwd = 2)  
    # 绘制Son和Father的回归线,使用线性回归模型
    abline(lm(Son ~ Father, data = height), lty = 1, lwd = 2)  
    
    # 查看线性回归的结果,显示回归系数等信息
    summary(lm(Son ~ Father, data = height))  
    
    # 读取另一份数据文件“reading.csv”,并指定列名
    reading <- read.csv("reading.csv", col.names = c("S1982", "S1983"))  
    # 绘制S1982和S1983的散点图
    plot(S1983 ~ S1982, data = reading, bty = "l", pch = 20)  
    
    # 计算S1983和S1982的均值
    S1983.ave = mean(reading$S1983)  
    S1982.ave = mean(reading$S1982)  
    # 绘制S1983和S1982的回归线,截距为S1983的均值减去S1982的均值,斜率为1
    abline(a = S1983.ave - S1982.ave, b = 1, lty = 1, lwd = 2)  
    # 绘制S1983和S1982的回归线,使用线性回归模型
    abline(lm(S1983 ~ S1982, data = reading), lty = 1, lwd = 2)  
    
    # 创建线性回归模型,预测S1983基于S1982
    model <- lm(S1983 ~ S1982, data = reading)  
    # 输出回归模型的摘要
    model  
    # 显示线性回归模型的详细统计信息
    summary(model)  
    
    # 设置显著性水平alpha为0.05
    alpha <- 0.05  
    # 计算回归系数的置信区间,置信度为1-alpha
    confint(model, level = 1 - alpha)  
    
    \end{lstlisting}


    \section*{12.04}

    \subsection*{回归分析}
    
    在两组数据上绘制回归线,使用 \texttt{lm()} 进行线性回归,计算回归系数,并生成回归模型的摘要。
    
    \subsection*{置信区间}
    
    通过 \texttt{confint()} 计算回归系数的置信区间,设定显著性水平为 0.05。



\chapter{12.09}

\begin{lstlisting}[language=R]
    # 安装并加载 deming 包(如果尚未安装,可以取消注释进行安装)
# install.packages('deming')  
library(deming)  # 加载deming包,用于Deming回归

# 绘制散点图,bty控制图的边框类型,pch控制点的形状
plot(aes ~ aas, data = arsenate, bty = "l", pch = 20)  

# 添加线性回归直线,使用 lm() 计算线性回归,并用 abline() 绘制回归线
abline(lm(aes ~ aas, data = arsenate), lty = 1, lwd = 2, col = 2)  

# 添加Deming回归直线,使用 deming() 函数计算Deming回归,并用 abline() 绘制回归线
abline(deming(aes ~ aas, data = arsenate, xstd = se.aas, ystd = se.aes), lty = 1, lwd = 2, col = 4)  

# 使用 lm() 建立线性回归模型,并存储为 fit1
fit1 = lm(aes ~ aas, data = arsenate)  

# 使用 deming() 建立Deming回归模型,并存储为 fit2
fit2 = deming(aes ~ aas, data = arsenate, xstd = se.aas, ystd = se.aes)  

# 打印线性回归的结果
print(fit1)  

# 打印Deming回归的结果
print(fit2)  

# 计算线性回归模型的斜率
slope_lm = coefficients(fit1)[2]  

# 计算Deming回归模型的斜率
slope_deming = coefficients(fit2)[2]  

# 计算两者斜率的差值
slope_lm - slope_deming    # -0.1283495  

\end{lstlisting}

\section*{解释}

\subsection*{安装和加载 deming 包}

\texttt{deming} 包提供了 Deming 回归分析工具,该方法是处理误差变量的回归分析。

\texttt{library(deming)}:加载该包,使得后续代码可以使用其中的函数。

\subsection*{绘制散点图}

\texttt{plot(aes ~ aas, data = arsenate, bty = "l", pch = 20)}:这行代码用于绘制 \texttt{aes} 和 \texttt{aas} 之间的散点图。\texttt{bty = "l"} 控制图形的边框样式,\texttt{pch = 20} 设置点的形状为实心圆。

\subsection*{添加回归线}

\texttt{abline(lm(aes ~ aas, data = arsenate), lty = 1, lwd = 2, col = 2)}:这行代码使用普通最小二乘法(OLS)拟合线性回归,并将回归直线添加到散点图中。\texttt{lty = 1} 为线型,\texttt{lwd = 2} 设置线条宽度,\texttt{col = 2} 设置颜色为红色。

\texttt{abline(deming(aes ~ aas, data = arsenate, xstd = se.aas, ystd = se.aes), lty = 1, lwd = 2, col = 4)}:这行代码使用 Deming 回归拟合,并将回归直线添加到散点图中,颜色设置为蓝色。

\subsection*{建立回归模型}

\texttt{fit1 = lm(aes ~ aas, data = arsenate)}:使用线性回归(OLS)建立 \texttt{aes} 对 \texttt{aas} 的回归模型。

\texttt{fit2 = deming(aes ~ aas, data = arsenate, xstd = se.aas, ystd = se.aes)}:使用 Deming 回归建立相同的回归模型。

\subsection*{打印回归结果}

\texttt{print(fit1)}:打印线性回归模型的结果。

\texttt{print(fit2)}:打印 Deming 回归模型的结果。

\subsection*{计算斜率差异}

\texttt{slope\_lm = coefficients(fit1)[2]}:提取线性回归模型的斜率。

\texttt{slope\_deming = coefficients(fit2)[2]}:提取 Deming 回归模型的斜率。

\texttt{slope\_lm - slope\_deming}:计算两个模型斜率的差异,结果为 -0.1283495,表明两种方法的回归斜率存在一定差异。

通过这段代码,你可以比较普通最小二乘法(OLS)回归和 Deming 回归在处理具有误差的自变量时的不同结果。



\chapter{12.25}

\begin{lstlisting}[language=R]
    # 1. 加载数据
brainsize = read.csv("C:/Users/78003/Desktop/brainsize.csv")  # 读取存储在桌面上的 CSV 文件,包含脑大小(BRAIN)、体重(BODY)、妊娠期(GESTATION)、胎数(LITTER)等数据

# 2. 先运行散点图矩阵(可选)
pairs(brainsize[, 2:5])  # 绘制数据集中第2至第5列(BODY、GESTATION、LITTER、BRAIN)之间的散点图矩阵,帮助检查变量之间的关系

# 3. 创建线性回归模型
linearmodel = lm(BRAIN ~ BODY + GESTATION + LITTER, data = brainsize)  # 创建一个线性回归模型,预测BRAIN(脑大小)基于BODY(体重)、GESTATION(妊娠期)和LITTER(胎数)

# 4. 查看回归模型摘要
summary(linearmodel)  # 输出线性回归模型的摘要,包括回归系数、标准误差、t值、p值等信息

# 5. 绘制BRAIN的密度图
plot(density(brainsize$BRAIN))  # 绘制脑大小(BRAIN)的密度图,帮助观察数据分布的形态

# 6. 绘制log(BRAIN)的密度图
plot(density(log(brainsize$BRAIN)))  # 绘制对数变换后的脑大小(log(BRAIN))的密度图,检查数据是否服从对数正态分布

# 7. Q-Q图,检查log(BRAIN)的正态性
qqnorm(log(brainsize$BRAIN))  # 绘制log(BRAIN)的Q-Q图,检查数据是否接近正态分布
qqline(log(brainsize$BRAIN))  # 添加Q-Q图的参考线,帮助判断数据的正态性

# 8. 对数变换后的散点图矩阵
pairs(cbind(log(brainsize[,2]), log(brainsize[,3]), log(brainsize[,4]), log(brainsize[,5])))  # 绘制对数变换后的散点图矩阵,检查log变换后的各个变量之间的关系

# 9. 使用对数变换创建新的线性回归模型
linearmodel = lm(log(BRAIN) ~ log(BODY) + log(GESTATION) + log(LITTER), data = brainsize)  # 创建新的线性回归模型,预测对数变换后的BRAIN基于对数变换后的BODY、GESTATION和LITTER

summary(linearmodel)  # 输出新的线性回归模型的摘要,检查回归系数和其他统计信息

# 10. 查看模型残差的Q-Q图
qqnorm(linearmodel$residuals)  # 绘制回归模型残差的Q-Q图,检查残差是否接近正态分布
qqline(linearmodel$residuals)  # 添加Q-Q图的参考线,帮助判断残差的正态性

\end{lstlisting}


\section*{解释}

\subsection*{加载数据}

\texttt{read.csv()} 用于读取 CSV 格式的文件,将数据加载到 R 中的 \texttt{brainsize} 数据框中。

\subsection*{散点图矩阵}

\texttt{pairs()} 用于生成一组变量之间的散点图矩阵。通过查看散点图矩阵,你可以初步了解变量之间的关系。

\subsection*{线性回归模型}

\texttt{lm()} 用于建立线性回归模型,指定自变量(\texttt{BODY}、\texttt{GESTATION}、\texttt{LITTER})和因变量(\texttt{BRAIN})。该模型假设脑大小是体重、妊娠期和胎数的线性函数。

\subsection*{回归模型摘要}

\texttt{summary()} 提供模型的详细信息,包括回归系数、标准误差、t 值、p 值等统计量。通过这些信息,你可以了解模型的拟合情况。

\subsection*{密度图}

\texttt{density()} 用于估计并绘制数据的密度图,帮助理解数据的分布情况。这里分别绘制了脑大小和其对数值的密度图。

\subsection*{Q-Q 图}

\texttt{qqnorm()} 用于绘制 Q-Q 图,以检查数据是否符合正态分布。\texttt{qqline()} 添加参考线,用于判断数据是否偏离正态分布。

\subsection*{对数变换后的散点图矩阵}

\texttt{cbind()} 函数将数据的对数变换合并在一起,\texttt{pairs()} 绘制对数变换后的散点图矩阵。对数变换有助于稳定数据的方差,并可能使数据分布更接近正态分布。

\subsection*{新的线性回归模型}

使用对数变换后的数据建立新的回归模型,帮助缓解原数据可能存在的非正态性或异方差问题。

\subsection*{残差 Q-Q 图}

通过分析模型残差的 Q-Q 图,检查残差是否服从正态分布,进而判断回归模型的适用性。

通过这些步骤,你可以全面地检查数据的分布、回归模型的拟合情况以及变换后数据的正态性。这些是进行线性回归分析和验证的重要步骤。



\chapter{终点}

\section*{知识点梳理与解释}

\subsection*{文档设置与宏包导入}

\begin{itemize}
    \item \texttt{documentclass[UTF8]\{report\}}:设定文档类型为报告,编码格式为 UTF-8。
    \item \texttt{usepackage\{...\}}:导入各种宏包,如 \texttt{ctex}(中文支持)、\texttt{amsmath}(数学公式)、\texttt{graphicx}(插图)、\texttt{listings}(代码高亮)等。
    \item \texttt{geometry\{...\}}:设置页面边距。
    \item \texttt{fancyhdr}:设置页眉页脚。
    \item \texttt{titlesec}:自定义标题格式。
\end{itemize}

\subsection*{数据读取与处理}

\begin{itemize}
    \item \texttt{read.csv()} 和 \texttt{read.table()}:读取 CSV 或 TXT 文件,将数据加载到 R 中的数据框中。例如:
    \begin{lstlisting}[language=R]
    brainsize = read.csv("C:/Users/78003/Desktop/brainsize.csv")
    \end{lstlisting}
    \item \texttt{setwd()}:设置工作目录,确保 R 能找到需要读取的文件。例如:
    \begin{lstlisting}[language=R]
    setwd("C:/Users/WangKe/Desktop/TA/2024TA/exercise class")
    \end{lstlisting}
    \item \texttt{getwd()}:获取当前工作目录。例如:
    \begin{lstlisting}[language=R]
    getwd()
    \end{lstlisting}
\end{itemize}

\subsection*{数据可视化}

\begin{itemize}
    \item \texttt{plot()}:绘制散点图,展示两个变量之间的关系。例如:
    \begin{lstlisting}[language=R]
    plot(Son ~ Father, data = height, bty = "l", pch = 20)
    \end{lstlisting}
    \item \texttt{abline()}:在散点图上添加回归线。例如:
    \begin{lstlisting}[language=R]
    abline(lm(Son ~ Father, data = height), lty = 1, lwd = 2)
    \end{lstlisting}
    \item \texttt{hist()}:绘制直方图,检查数据分布。例如:
    \begin{lstlisting}[language=R]
    hist(height$Father)
    \end{lstlisting}
    \item \texttt{density()}:绘制密度图,估计数据分布。例如:
    \begin{lstlisting}[language=R]
    plot(density(height$Father))
    \end{lstlisting}
    \item \texttt{qqnorm()} 和 \texttt{qqline()}:绘制 Q-Q 图,检查数据是否符合正态分布。例如:
    \begin{lstlisting}[language=R]
    qqnorm(height$Father)
    qqline(height$Father)
    \end{lstlisting}
    \item \texttt{pairs()}:生成散点图矩阵,展示多个变量之间的关系。例如:
    \begin{lstlisting}[language=R]
    pairs(brainsize[, 2:5])
    \end{lstlisting}
    \item \texttt{bwplot()}:绘制箱线图,展示数据分布。例如:
    \begin{lstlisting}[language=R]
    bwplot(Expt ~ Speed, data = michelson, main = "Speed of Light Data", ylab = "Experiment No.")
    \end{lstlisting}
\end{itemize}

\subsection*{统计检验与回归分析}

\begin{itemize}
    \item \texttt{lm()}:建立线性回归模型,指定自变量和因变量。例如:
    \begin{lstlisting}[language=R]
    linearmodel = lm(BRAIN ~ BODY + GESTATION + LITTER, data = brainsize)
    \end{lstlisting}
    \item \texttt{summary()}:提供模型的详细信息,包括回归系数、标准误差、t 值、p 值等统计量。例如:
    \begin{lstlisting}[language=R]
    summary(linearmodel)
    \end{lstlisting}
    \item \texttt{confint()}:计算回归系数的置信区间。例如:
    \begin{lstlisting}[language=R]
    confint(model, level = 1 - alpha)
    \end{lstlisting}
    \item \texttt{z.test()}:使用 \texttt{BSDA} 包进行 Z 检验。例如:
    \begin{lstlisting}[language=R]
    result = z.test(nail, mu = mu, alternative = c("two.sided"), sigma.x = sigma)
    \end{lstlisting}
    \item \texttt{t.test()}:进行 t 检验。例如:
    \begin{lstlisting}[language=R]
    result = t.test(nail, mu = mu, alternative = c("two.sided"))
    \end{lstlisting}
    \item \texttt{wilcox.test()}:进行 Wilcoxon 秩和检验。例如:
    \begin{lstlisting}[language=R]
    result = wilcox.test(Aarea, Barea, exact = TRUE, alternative = c("two.sided"))
    \end{lstlisting}
    \item \texttt{chisq.test()}:进行卡方检验。例如:
    \begin{lstlisting}[language=R]
    result = chisq.test(table1[, 1:3])
    \end{lstlisting}
    \item \texttt{deming()}:进行 Deming 回归,处理误差变量的回归分析。例如:
    \begin{lstlisting}[language=R]
    fit2 = deming(aes ~ aas, data = arsenate, xstd = se.aas, ystd = se.aes)
    \end{lstlisting}
\end{itemize}

\subsection*{自助法(Bootstrapping)}

\begin{itemize}
    \item 通过对样本进行有放回抽样,再次计算估计值,并计算这些估计值的分布,用于估计统计量的置信区间。例如:
    \begin{lstlisting}[language=R]
    bootstrapestimates = rep(0, bootstrapsize)
    for (ii in 1:bootstrapsize) {
        bootstrapsample = rexp(mysamplesize, rate = 1) + mymle
        bootstrapestimates[ii] = sort(bootstrapsample)[1]
    }
    bootstrapquantiles = sort(bootstrapestimates - mymle)
    lowerquantile = bootstrapquantiles[round(bootstrapsize*alpha*0.5)]
    upperquantile = bootstrapquantiles[round(bootstrapsize*(1-alpha*0.5))]
    lowerbound = mymle - upperquantile
    upperbound = mymle - lowerquantile
    \end{lstlisting}
    \item 计算置信区间的上下界,计算方式是将分位数从原始估计值中减去。例如:
    \begin{lstlisting}[language=R]
    lowerbound = mymomentestimation - upperquantile
    upperbound = mymomentestimation - lowerquantile
    \end{lstlisting}
\end{itemize}

\subsection*{数据变换与分布检验}

\begin{itemize}
    \item 对数变换:使用 \texttt{log()} 函数对数据进行对数变换,稳定数据的方差,使数据分布更接近正态分布。例如:
    \begin{lstlisting}[language=R]
    plot(density(log(brainsize$BRAIN)))
    \end{lstlisting}
    \item \texttt{stem()}:绘制茎叶图,展示数据分布。例如:
    \begin{lstlisting}[language=R]
    stem(hills$time)
    \end{lstlisting}
    \item \texttt{splom()}:绘制散点图矩阵,展示数据中各个变量的关系。例如:
    \begin{lstlisting}[language=R]
    splom(~ hills)
    \end{lstlisting}
    \item \texttt{attach()} 和 \texttt{detach()}:使数据集中的变量可以直接使用,避免变量名冲突。例如:
    \begin{lstlisting}[language=R]
    attach(hills)
    detach(hills)
    \end{lstlisting}
\end{itemize}

\subsection*{模型残差分析}

\begin{itemize}
    \item 通过分析模型残差的 Q-Q 图,检查残差是否服从正态分布,进而判断回归模型的适用性。例如:
    \begin{lstlisting}[language=R]
    qqnorm(linearmodel$residuals)
    qqline(linearmodel$residuals)
    \end{lstlisting}
\end{itemize}

\subsection*{示例代码与解释}

\begin{itemize}
    \item 通过具体的 R 代码示例,展示如何进行数据读取、可视化、统计检验、回归分析、自助法等操作。
    \item 对每段代码进行详细解释,帮助理解每个步骤的目的和实现方法。
\end{itemize}

\subsection*{总结}

通过这些步骤和知识点的梳理,你可以全面地进行数据分析和回归建模,包括数据读取、可视化、统计检验、自助法、数据变换、分布检验和模型残差分析等。这些是进行数据分析和验证的重要步骤。








\end{document}